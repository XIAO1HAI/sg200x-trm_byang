%% Generated by Sphinx.
\def\sphinxdocclass{report}
\documentclass[letterpaper,10pt,english]{sphinxmanual}
\ifdefined\pdfpxdimen
   \let\sphinxpxdimen\pdfpxdimen\else\newdimen\sphinxpxdimen
\fi \sphinxpxdimen=.75bp\relax
\ifdefined\pdfimageresolution
    \pdfimageresolution= \numexpr \dimexpr1in\relax/\sphinxpxdimen\relax
\fi
%% let collapsible pdf bookmarks panel have high depth per default
\PassOptionsToPackage{bookmarksdepth=5}{hyperref}
%% turn off hyperref patch of \index as sphinx.xdy xindy module takes care of
%% suitable \hyperpage mark-up, working around hyperref-xindy incompatibility
\PassOptionsToPackage{hyperindex=false}{hyperref}
%% memoir class requires extra handling
\makeatletter\@ifclassloaded{memoir}
{\ifdefined\memhyperindexfalse\memhyperindexfalse\fi}{}\makeatother

\PassOptionsToPackage{booktabs}{sphinx}
\PassOptionsToPackage{colorrows}{sphinx}

\PassOptionsToPackage{warn}{textcomp}

\catcode`^^^^00a0\active\protected\def^^^^00a0{\leavevmode\nobreak\ }
\usepackage{cmap}
\usepackage{xeCJK}
\usepackage{amsmath,amssymb,amstext}
\usepackage{babel}



\setmainfont{FreeSerif}[
  Extension      = .otf,
  UprightFont    = *,
  ItalicFont     = *Italic,
  BoldFont       = *Bold,
  BoldItalicFont = *BoldItalic
]
\setsansfont{FreeSans}[
  Extension      = .otf,
  UprightFont    = *,
  ItalicFont     = *Oblique,
  BoldFont       = *Bold,
  BoldItalicFont = *BoldOblique,
]
\setmonofont{FreeMono}[
  Extension      = .otf,
  UprightFont    = *,
  ItalicFont     = *Oblique,
  BoldFont       = *Bold,
  BoldItalicFont = *BoldOblique,
]



\usepackage[Sonny]{fncychap}
\ChNameVar{\Large\normalfont\sffamily}
\ChTitleVar{\Large\normalfont\sffamily}
\usepackage[,numfigreset=1,mathnumfig]{sphinx}

\fvset{fontsize=\small,formatcom=\xeCJKVerbAddon}
\usepackage{geometry}


% Include hyperref last.
\usepackage{hyperref}
% Fix anchor placement for figures with captions.
\usepackage{hypcap}% it must be loaded after hyperref.
% Set up styles of URL: it should be placed after hyperref.
\urlstyle{same}

\addto\captionsenglish{\renewcommand{\contentsname}{目录}}

\usepackage{sphinxmessages}
\setcounter{tocdepth}{3}
\setcounter{secnumdepth}{3}

		\usepackage{tocloft}
		\usepackage[document]{ragged2e}
		\newcommand{\sectionbreak}{\clearpage}
		\renewcommand\cftfignumwidth{4em}
		\renewcommand\cfttabnumwidth{4em}
		\renewcommand\cftsecnumwidth{4em}
		\renewcommand\cftsubsecnumwidth{6em}
		\renewcommand\cftparanumwidth{6em}
		\usepackage{fancyhdr}
		\setlength\headheight{14pt}
		\fancypagestyle{normal}{
			\fancyhead[R]{}
			\fancyhead[C]{\leftmark}
			\fancyfoot[C]{Copyright © 2023 SOPHGO Co., Ltd}
			\fancyfoot[R]{\thepage}
			\renewcommand{\headrulewidth}{0.4pt}
			\renewcommand{\footrulewidth}{0pt}
		}

\title{SG2000}
\date{2023 年 12 月 14 日}
\release{1.0\sphinxhyphen{}alpha}
\author{Sophgo}
\newcommand{\sphinxlogo}{\vbox{}}
\renewcommand{\releasename}{发行版本}
\makeindex
\begin{document}

\ifdefined\shorthandoff
  \ifnum\catcode`\=\string=\active\shorthandoff{=}\fi
  \ifnum\catcode`\"=\active\shorthandoff{"}\fi
\fi

\pagestyle{empty}

		\begin{titlepage}
		\begin{center}
		\includegraphics[width=0.4\textwidth]{SOPHON-LOGO.png}
		\vspace*{2cm}

		\Huge \textbf{SG2000} \par
		\vspace*{1cm}
		\Huge \textbf{技术参考手册} \par
		\vspace*{4cm}
		\end{center}
		\noindent \Large 版本 : 1.0-alpha\par
		\noindent \Large 发布日期 : 2023-11-22\par
		\vspace*{2cm}
		\noindent \normalsize Copyright © 2023 SOPHGO Co., Ltd. All rights reserved.\\
		\end{titlepage}
\pagestyle{plain}
\sphinxtableofcontents
\pagestyle{normal}
\phantomsection\label{\detokenize{index::doc}}


\sphinxAtStartPar
\sphinxstylestrong{修订记录}


\begin{savenotes}\sphinxattablestart
\sphinxthistablewithglobalstyle
\centering
\begin{tabulary}{\linewidth}[t]{TTT}
\sphinxtoprule
\sphinxstyletheadfamily 
\sphinxAtStartPar
Revision
&\sphinxstyletheadfamily 
\sphinxAtStartPar
Date
&\sphinxstyletheadfamily 
\sphinxAtStartPar
Description
\\
\sphinxmidrule
\sphinxtableatstartofbodyhook
\sphinxAtStartPar
1.0\sphinxhyphen{}alpha
&
\sphinxAtStartPar
2023/12/13
&
\sphinxAtStartPar
初稿
\\
\sphinxbottomrule
\end{tabulary}
\sphinxtableafterendhook\par
\sphinxattableend\end{savenotes}

\sphinxstepscope


\chapter{声明 for SG2000}
\label{\detokenize{contents/disclaimer:for-sg2000}}\label{\detokenize{contents/disclaimer::doc}}
\begin{figure}[htbp]
\centering

\noindent\sphinxincludegraphics{{SOPHON-LOGO}.png}
\end{figure}

\begin{DUlineblock}{0em}
\item[] \sphinxstylestrong{法律声明}
\item[] 本数据手册包含北京晶视智能科技有限公司(下称“晶视智能”)的保密信息。未经授权,禁止使用或披露本数据手册中包含的信息。如您未经授权披露全部或部分保密信息,导致晶视智能遭受任何损失或损害,您应对因之产生的损失/损害承担责任。
\item[] 
\item[] 本文件内信息如有更改,恕不另行通知。晶视智能不对使用或依赖本文件所含信息承担任何责任。
\item[] 
\item[] 本数据手册和本文件所含的所有信息均按“原样”提供,无任何明示、暗示、法定或其他形式的保证。晶视智能特别声明未做任何适销性、非侵权性和特定用途适用性的默示保证,亦对本数据手册所使用、包含或提供的任何第三方的软件不提供任何保证;用户同意仅向该第三方寻求与此相关的任何保证索赔。此外,晶视智能亦不对任何其根据用户规格或符合特定标准或公开讨论而制作的可交付成果承担责任。
\end{DUlineblock}

\begin{DUlineblock}{0em}
\item[] \sphinxstylestrong{联系我们}
\end{DUlineblock}
\begin{quote}\begin{description}
\sphinxlineitem{地址}
\sphinxAtStartPar
北京市海淀区丰豪东路9号院中关村集成电路设计园(ICPARK)1号楼  深圳市宝安区福海街道展城社区会展湾云岸广场T10栋

\sphinxlineitem{电话}
\sphinxAtStartPar
+86\sphinxhyphen{}10\sphinxhyphen{}57590723  +86\sphinxhyphen{}10\sphinxhyphen{}57590724

\sphinxlineitem{邮编}
\sphinxAtStartPar
100094(北京) 518100(深圳)

\sphinxlineitem{官方网站}
\sphinxAtStartPar
\sphinxurl{https://www.sophgo.com/}

\sphinxlineitem{技术论坛}
\sphinxAtStartPar
\sphinxurl{https://developer.sophgo.com/forum/index.html}

\end{description}\end{quote}

\sphinxstepscope


\chapter{系统概述}
\label{\detokenize{contents/system-overview/0.index:id1}}\label{\detokenize{contents/system-overview/0.index::doc}}
\sphinxstepscope


\section{介绍}
\label{\detokenize{contents/system-overview/introduction:id1}}\label{\detokenize{contents/system-overview/introduction::doc}}
\sphinxAtStartPar
SG2000 是面向 AIot 领域推出的高性能、低功耗芯片,内置多个强劲内核:2 x C906、1 x Arm Cortex A53、1 x 8051,用户可以根据需求快速切换内核。同时集成硬件编解码器:H.264 视频压缩编解码器, H.265 视频压缩编码器和 ISP,支持配置了专业级的视频图像 ISP:HDR宽动态、3D降噪、除雾、镜头畸变校正等多种图像增强和矫正算法。

\sphinxAtStartPar
芯片更集成了自研 TPU,可提供0.5TOPS IN8 的算力。 特殊设计的 TPU 调度引擎能有效地为所有的张量处理器核心提供极高的带宽数据流。

\sphinxAtStartPar
同时支持丰富的外设接口:6 x I2C、2 x SDIO3.0、2 x I2S、16 x PWM、1 x USB 2.0 等,满足多种场景需求。

\sphinxstepscope


\section{系统框架图}
\label{\detokenize{contents/system-overview/system-block-diagram:id1}}\label{\detokenize{contents/system-overview/system-block-diagram::doc}}
\begin{figure}[htbp]
\centering
\capstart

\noindent\sphinxincludegraphics{{image_1}.jpg}
\caption{系统框架}\label{\detokenize{contents/system-overview/system-block-diagram:id2}}\label{\detokenize{contents/system-overview/system-block-diagram:diagram-system-block}}\end{figure}

\sphinxstepscope


\section{特性介绍}
\label{\detokenize{contents/system-overview/features:id1}}\label{\detokenize{contents/system-overview/features::doc}}

\subsection{处理器内核}
\label{\detokenize{contents/system-overview/features:id2}}\begin{itemize}
\item {} 
\sphinxAtStartPar
主处理器 RISCV C906 @ 1.0Ghz
\begin{itemize}
\item {} 
\sphinxAtStartPar
32KB I\sphinxhyphen{}cache, 64KB D\sphinxhyphen{}Cache

\item {} 
\sphinxAtStartPar
集成矢量(Vector)及浮点运算单元 (FPU)

\end{itemize}

\item {} 
\sphinxAtStartPar
主处理器 ARM Cortex\sphinxhyphen{}A53 @ 1.0Ghz
\begin{itemize}
\item {} 
\sphinxAtStartPar
32KB I\sphinxhyphen{}cache, 32KB D\sphinxhyphen{}Cache

\item {} 
\sphinxAtStartPar
128KB L2 cache

\item {} 
\sphinxAtStartPar
支持 Neon 以及浮点运算 FPU

\end{itemize}

\item {} 
\sphinxAtStartPar
协处理器 RISCV C906 @ 700Mhz
\begin{itemize}
\item {} 
\sphinxAtStartPar
集成浮点运算单元 (FPU)

\end{itemize}

\end{itemize}


\subsection{TPU}
\label{\detokenize{contents/system-overview/features:tpu}}\begin{itemize}
\item {} 
\sphinxAtStartPar
内建 CVITEK TPU , 算力达到 \textasciitilde{}0.5TOPS INT8

\item {} 
\sphinxAtStartPar
支持主流的神经网络架构: Caffe,Pytorch,
TensorFlow(Lite),ONNX和MXNet

\item {} 
\sphinxAtStartPar
可实現行人侦测 (Pedestrian Detection) , 人脸侦测 (Face Detection) ,
人脸识别 (Face recognition) , 活体侦测 (Face anti\sphinxhyphen{}spoofing)
及其他视频结构化应用。

\end{itemize}


\subsection{视频编解码}
\label{\detokenize{contents/system-overview/features:id3}}\begin{itemize}
\item {} 
\sphinxAtStartPar
H.264 Baseline/Main/High profile

\item {} 
\sphinxAtStartPar
H.265 Main profile

\item {} 
\sphinxAtStartPar
H.264/H.265 均支援 I 帧及 P 帧

\item {} 
\sphinxAtStartPar
MJPEG/JPEG baseline

\item {} 
\sphinxAtStartPar
H.264编解码最大分辨率 : 2880x1620 (5M)

\item {} 
\sphinxAtStartPar
H.265 编码最大分辨率 : 2880x1620 (5M)

\item {} 
\sphinxAtStartPar
H.264 编解码性能
\begin{itemize}
\item {} 
\sphinxAtStartPar
2880x1620@30fps+720x576@30fps

\item {} 
\sphinxAtStartPar
\sphinxhref{mailto:1920x1080@30fps}{1920x1080@30fps} 编码 + \sphinxhref{mailto:1920x1080@30fps}{1920x1080@30fps} 解码

\end{itemize}

\item {} 
\sphinxAtStartPar
H.265 编码性能
\begin{itemize}
\item {} 
\sphinxAtStartPar
2880x1620@30fps+720x576@30fps

\end{itemize}

\item {} 
\sphinxAtStartPar
JPEG 最大编解码性能
\begin{itemize}
\item {} 
\sphinxAtStartPar
\sphinxhref{mailto:2880x1620@30fps}{2880x1620@30fps}

\end{itemize}

\item {} 
\sphinxAtStartPar
支持 CBR/VBR/FIXQP 等多种码率控制模式

\item {} 
\sphinxAtStartPar
支持 感兴趣区域 (ROI) 编码

\end{itemize}


\subsection{视频接口}
\label{\detokenize{contents/system-overview/features:id4}}\begin{itemize}
\item {} 
\sphinxAtStartPar
输入

\item {} 
\sphinxAtStartPar
支持同时兩路视频输入(mipi 2L+2L + DVP or MIPI 4L)

\item {} 
\sphinxAtStartPar
支持 MIPI, Sub\sphinxhyphen{}LVDS, HiSPI 等串行接口

\item {} 
\sphinxAtStartPar
支持 8/10/12 bit RGB Bayer 视频输入

\item {} 
\sphinxAtStartPar
支持 BT.601,BT.656,BT.1120     视频输入

\item {} 
\sphinxAtStartPar
支持AHD多路混合 BT格式

\item {} 
\sphinxAtStartPar
支持 SONY, OnSemi, OmniVision 等高清 CMOS sensor

\item {} 
\sphinxAtStartPar
提供可编程频率输出供 sensor 作为参考时钟

\item {} 
\sphinxAtStartPar
支持最大宽度为 2880 , 最大分辨率 5M (2688x1944, 2880x1620)

\item {} 
\sphinxAtStartPar
输出
\begin{itemize}
\item {} 
\sphinxAtStartPar
支持多种串行与并行屏显规格

\item {} 
\sphinxAtStartPar
支持 MIPI等串行接口

\item {} 
\sphinxAtStartPar
支持 支持 BT.601, BT.656, BT.1120, RGB565/666/888, 8080等并行输出接口

\item {} 
\sphinxAtStartPar
支持 SPI 输出接口

\end{itemize}

\end{itemize}


\subsection{ISP与图像处理}
\label{\detokenize{contents/system-overview/features:isp}}\begin{itemize}
\item {} 
\sphinxAtStartPar
图像视频90度、180度、270度旋转

\item {} 
\sphinxAtStartPar
图像视频Mirror、Flip功能

\item {} 
\sphinxAtStartPar
视频2层OSD叠加

\item {} 
\sphinxAtStartPar
视频1/32~32x缩放功能

\item {} 
\sphinxAtStartPar
3A(AE/AWB/AF)算法

\item {} 
\sphinxAtStartPar
固定模式噪声消除、坏点校正

\item {} 
\sphinxAtStartPar
镜头阴影校正、镜头畸变校正、紫边校正

\item {} 
\sphinxAtStartPar
方向自适应demosaic

\item {} 
\sphinxAtStartPar
Gamma校正、(区域/全域)动态对比度增强、颜色管理和增强

\item {} 
\sphinxAtStartPar
区域自适应去雾

\item {} 
\sphinxAtStartPar
Bayer降噪、3D降噪、细节增强及锐化增强

\item {} 
\sphinxAtStartPar
Local Tone mapping

\item {} 
\sphinxAtStartPar
Sensor自带宽动态和2帧宽动态

\item {} 
\sphinxAtStartPar
两轴数字图像防抖

\item {} 
\sphinxAtStartPar
镜头畸变校正

\item {} 
\sphinxAtStartPar
提供PC 端ISP tuning tools

\end{itemize}


\subsection{硬件加速引擎}
\label{\detokenize{contents/system-overview/features:id5}}\begin{itemize}
\item {} 
\sphinxAtStartPar
软硬体混合模式支持部分 OpenCV 库

\item {} 
\sphinxAtStartPar
软硬体混合模式支持部分 IVE 库

\end{itemize}


\subsection{音频编解码}
\label{\detokenize{contents/system-overview/features:id6}}\begin{itemize}
\item {} 
\sphinxAtStartPar
集成 Audio CODEC, 支持 16 bit 音源/语音 输入和输出

\item {} 
\sphinxAtStartPar
集成双声道麦克风输入

\item {} 
\sphinxAtStartPar
集成双声道输出 (需要外挂功放才能推动喇叭)

\item {} 
\sphinxAtStartPar
同时支持以 I2S/PCM/TDM    接口连接外部 audio    CODEC,  内建 audio        PLL     支持 MCLK 输出

\item {} 
\sphinxAtStartPar
软件音频编解码协议 (G.711, G.726, ADPCM)

\item {} 
\sphinxAtStartPar
软件支持音频 3A (AEC, ANR, AGC) 功能

\end{itemize}


\subsection{网络接口}
\label{\detokenize{contents/system-overview/features:id7}}\begin{itemize}
\item {} 
\sphinxAtStartPar
以太网模块提供1个 Ethernet MAC , 实现网路数据的接收与发送。

\item {} 
\sphinxAtStartPar
Ethernet MAC 搭配内建10/100Mbps Fast Ethernet Transceiver
可工作在10/100Mbps 全双工或半双工模式,也可通过 RMII 外挂 PHY。

\end{itemize}


\subsection{安全系统模块}
\label{\detokenize{contents/system-overview/features:id8}}\begin{itemize}
\item {} 
\sphinxAtStartPar
硬件实现AES/DES/SM4多种加解密算法

\item {} 
\sphinxAtStartPar
硬件实现HASH(SHA1/SHA256) 哈希算法

\item {} 
\sphinxAtStartPar
硬件实现随机数发生器

\item {} 
\sphinxAtStartPar
内部集成2Kbit eFuse逻辑空间

\end{itemize}


\subsection{智能安全运行环境}
\label{\detokenize{contents/system-overview/features:id9}}\begin{itemize}
\item {} 
\sphinxAtStartPar
支持信任链建立:
提供安全环境的基础,为可信环境的根本,如硬件安全设置、信任根

\item {} 
\sphinxAtStartPar
支持安全启动,提供安全硬件、软件保护功能

\item {} 
\sphinxAtStartPar
支持资料加密安全: 数据加密程序,运算核心加密

\item {} 
\sphinxAtStartPar
支持软、固件验证流程:确认软件可信性及完整性,包括 开机及载入验证程序

\item {} 
\sphinxAtStartPar
支持安全储存及传输:保护外部数据储存及交换

\item {} 
\sphinxAtStartPar
支持安全更新

\end{itemize}


\subsection{外围接口}
\label{\detokenize{contents/system-overview/features:id10}}\begin{itemize}
\item {} 
\sphinxAtStartPar
集成POR, Power sequence

\item {} 
\sphinxAtStartPar
6 个单端ADC (3 no die domain)

\item {} 
\sphinxAtStartPar
6 个 I2C (1 no die domain)

\item {} 
\sphinxAtStartPar
4 个SPI

\item {} 
\sphinxAtStartPar
5 组 UART (1 no die domain)

\item {} 
\sphinxAtStartPar
4 组(16通道) PWM

\item {} 
\sphinxAtStartPar
2个SDIO接口

\item {} 
\sphinxAtStartPar
一个支援 3V 连接 SD 3.0 Card(支持最大容量SDXC 2TB, 支持速度为UHS\sphinxhyphen{}I)

\item {} 
\sphinxAtStartPar
一个支援 1.8V/3.0V 连接其他SDIO 3.0设备.(支持速度为UHS\sphinxhyphen{}I)

\item {} 
\sphinxAtStartPar
110 GPIO 接口 (14 no die domain)

\item {} 
\sphinxAtStartPar
集成 keyscan 及 Wiegand

\item {} 
\sphinxAtStartPar
集成 MAC PHY 支援 10/100Mbps 全双工或半双工模式

\item {} 
\sphinxAtStartPar
一个 USB Host / device 接口

\end{itemize}


\subsection{外部存储器接口}
\label{\detokenize{contents/system-overview/features:id11}}\begin{itemize}
\item {} 
\sphinxAtStartPar
内建 DRAM
\begin{itemize}
\item {} 
\sphinxAtStartPar
DDR3 16bitx1, 最高速率达 1866Mbps, 容量 4Gbit (512MB)

\end{itemize}

\item {} 
\sphinxAtStartPar
SPI NOR flash 接口 (1.8V / 3.0V)
\begin{itemize}
\item {} 
\sphinxAtStartPar
支持 1, 2, 4 线模式

\item {} 
\sphinxAtStartPar
最大支持 256MByte

\end{itemize}

\item {} 
\sphinxAtStartPar
SPI Nand flash 接口 (1.8V / 3.0V)
\begin{itemize}
\item {} 
\sphinxAtStartPar
支持 1KB/2KB/4KB page (对应的最大容量 16GB/32GB/64GB)

\item {} 
\sphinxAtStartPar
使用器件本身内建的ECC模块

\end{itemize}

\item {} 
\sphinxAtStartPar
eMMC 4.5 接口 (1.8V/3.0V) SD0 EMMC 共電. 因為 SD卡 default 3V, 所以有
SD 卡时, 不适合接 1.8V eMMC。
\begin{itemize}
\item {} 
\sphinxAtStartPar
4 bit 接口

\item {} 
\sphinxAtStartPar
支持 HS200

\item {} 
\sphinxAtStartPar
最大支持容量 2TB

\end{itemize}

\end{itemize}


\subsection{芯片物理规格}
\label{\detokenize{contents/system-overview/features:id12}}\begin{itemize}
\item {} 
\sphinxAtStartPar
功耗
\begin{itemize}
\item {} 
\sphinxAtStartPar
1080P + Video encode + AI : \textasciitilde{} 500mW

\item {} 
\sphinxAtStartPar
其余场景 : TBD

\end{itemize}

\item {} 
\sphinxAtStartPar
工作电压
\begin{itemize}
\item {} 
\sphinxAtStartPar
内核电压为 0.9V

\item {} 
\sphinxAtStartPar
IO 电压为 1.8V 及 3.0V

\item {} 
\sphinxAtStartPar
DDR 电压
\begin{itemize}
\item {} 
\sphinxAtStartPar
1.35V

\end{itemize}

\end{itemize}

\item {} 
\sphinxAtStartPar
封装
\begin{itemize}
\item {} 
\sphinxAtStartPar
使用 LFBGA 封装, 封装尺寸为 10mmx10mmx1.3mm,管脚间距为 0.65mm,管脚总数为 205 个。

\end{itemize}

\end{itemize}



\renewcommand{\indexname}{索引}
\printindex
\end{document}